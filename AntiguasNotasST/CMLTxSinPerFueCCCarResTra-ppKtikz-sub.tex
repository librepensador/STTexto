\documentclass[ST]{subfiles}

\begin{document}
\usetikzlibrary{arrows}
\begin{figure}[h]
\centering
\begin{tikzpicture}[scale=2.54]
% dpic version 2010.11.28 option -g for TikZ and PGF 1.01

%\draw[step=.25cm,lightgray,very thin] (-0.25,-3) grid (4,1);
\draw [dashed] (0.5,0.75) -- (0.5,-2.75);
\draw  [dashed] (3.5,0.75) -- (3.5,-2.75);
\draw [-latex,thick] (0.375,-0.75) -- (3.625,-0.75) node[anchor=west] {$x$} ; % primer eje X
\draw [-latex,thick] (3.5,-0.875) -- (3.5,-0.125) node[anchor=north west] {$v$};      % primer eje v
\shadedraw[top color=white,bottom color=gray!70!white, rounded corners=0.1cm,shorten = 0.5pt] (0.5,-0.75)--(0.5,-0.6) -- (2.5,-0.6)--(2.6,-0.75)--cycle;
\draw[blue] (0.5,-0.25) -- (3.5,-0.25) node[midway,sloped,below,blue] {$v_\infty$} ;% primer V_0
\node at (0.25,-0.5) {$v_{+1}$};

\draw [-latex,thick] (0.375,-1.75) -- (3.625,-1.75) node[anchor=west] {$x$} ; % segundo eje X
\draw [-latex,thick] (3.5,-1.875) -- (3.5,-1.125) node[anchor=north west] {$v$} ;% segundo eje v
\shadedraw[top color=white,bottom color=gray!70!white, rounded corners=0.1cm,shorten =0.5pt] (0.5,-1.75)--(0.5,-1.6) -- (3.5,-1.6)--(3.5,-1.75)--cycle;
\shadedraw[top color=white,bottom color=gray!70!white, rounded corners=0.1cm,shorten =0.5pt] (3.5,-1.6)--(3.5,-1.475) -- (1.5,-1.475)--(1.4,-1.6)--cycle;
\draw[blue] (0.5,-1.25) -- (3.5,-1.25)  node[midway,sloped,above,blue] {$v_\infty$}  ;% segundo V_0
\node at (0.15,-1.5) {$v_{+1}+v_{-1}$};

\draw [-latex,thick] (0.375,-2.75) -- (3.625,-2.75) node[anchor=west] {$x$} ; % tercer eje X
\draw [-latex,thick] (3.5,-2.875) -- (3.5,-2.125) node[anchor=north west] {$v$} ;% tercer eje v
\shadedraw[top color=white,bottom color=gray!70!white, rounded corners=0.1cm,shorten =0.5pt] (0.5,-2.75)--(0.5,-2.475) -- (3.5,-2.475)--(3.5,-2.75)--cycle;
\shadedraw[top color=white,bottom color=gray!70!white, rounded corners=0.1cm,shorten =0.5pt]
(0.5,-2.475)--(0.5,-2.375) -- (2.5,-2.375)--(2.6,-2.475)--cycle;
\draw[blue] (0.5,-2.25) -- (3.5,-2.25) node[midway,sloped,above,blue] {$v_\infty$} ;% tercer V_0
\node at (0.15,-2.5) {$v_1+v_{+2}$};

\ifx\dpiclw\undefined\newdimen\dpiclw\fi
\global\def\dpicdraw{\draw[line width=\dpiclw]}
\global\def\dpicstop{;}
\dpiclw=0.8bp
\dpiclw=0.8bp
\dpicdraw (0,0)
 --(0,0.3125)\dpicstop
\dpicdraw (-0.0625,0.3125)
 --(0.0625,0.3125)\dpicstop
\dpicdraw (-0.125,0.354167)
 --(0.125,0.354167)\dpicstop
\dpicdraw (-0.0625,0.395833)
 --(0.0625,0.395833)\dpicstop
\dpicdraw (-0.125,0.4375)
 --(0.125,0.4375)\dpicstop
\dpicdraw (0,0.4375)
 --(0,0.75)\dpicstop
\draw (-0.125,0.375) node[left=-1.5bp]{$ V_g$};
\dpicdraw (0,0.75)
 --(0.125,0.75)
 --(0.145833,0.791667)
 --(0.1875,0.708333)
 --(0.229167,0.791667)
 --(0.270833,0.708333)
 --(0.3125,0.791667)
 --(0.354167,0.708333)
 --(0.375,0.75)
 --(0.5,0.75)\dpicstop
\draw (0.25,0.791667) node[above=-1.5bp]{$ R_g$};
\dpicdraw[fill=black](0.5,0.75) circle (0.007874in)\dpicstop
\dpicdraw[fill=black](0.5,0) circle (0.007874in)\dpicstop
\dpicdraw (2.858333,0.75)
 --(3.5,0.75)\dpicstop
\dpicdraw (0.5,0.75)
 --(1.166667,0.75)\dpicstop
\dpicdraw (1.166667,0.7)
 --(2.833333,0.7)\dpicstop
\dpicdraw (2.833333,0.7)
 ..controls (2.847141,0.7) and (2.858333,0.722385)
 ..(2.858333,0.75)
 ..controls (2.858333,0.777615) and (2.847141,0.8)
 ..(2.833333,0.8)\dpicstop
\dpicdraw (2.833333,0.8)
 --(1.166667,0.8)\dpicstop
\dpicdraw (1.166667,0.8)
 ..controls (1.152859,0.8) and (1.141667,0.777615)
 ..(1.141667,0.75)
 ..controls (1.141667,0.722385) and (1.152859,0.7)
 ..(1.166667,0.7)
 ..controls (1.180474,0.7) and (1.191667,0.722385)
 ..(1.191667,0.75)
 ..controls (1.191667,0.777615) and (1.180474,0.8)
 ..(1.166667,0.8)\dpicstop
\dpicdraw[fill=black](3.5,0.75) circle (0.007874in)\dpicstop
\dpicdraw[fill=black](3.5,0) circle (0.007874in)\dpicstop
\dpicdraw (3.5,0.75)
 --(3.75,0.75)\dpicstop
\dpicdraw (3.75,0.75)
 --(3.75,0.5)
 --(3.791667,0.479167)
 --(3.708333,0.4375)
 --(3.791667,0.395833)
 --(3.708333,0.354167)
 --(3.791667,0.3125)
 --(3.708333,0.270833)
 --(3.75,0.25)
 --(3.75,0)\dpicstop
\draw (3.791667,0.375) node[right=-1.5bp]{$ R_l$};
\dpicdraw (3.75,0)
 --(3.5,0)\dpicstop
\dpicdraw[fill=black](3.5,0) circle (0.007874in)\dpicstop
\dpicdraw (2.858333,0)
 --(3.5,0)\dpicstop
\dpicdraw (0.5,0)
 --(1.166667,0)\dpicstop
\dpicdraw (1.166667,-0.05)
 --(2.833333,-0.05)\dpicstop
\dpicdraw (2.833333,-0.05)
 ..controls (2.847141,-0.05) and (2.858333,-0.027615)
 ..(2.858333,0)
 ..controls (2.858333,0.027615) and (2.847141,0.05)
 ..(2.833333,0.05)\dpicstop
\dpicdraw (2.833333,0.05)
 --(1.166667,0.05)\dpicstop
\dpicdraw (1.166667,0.05)
 ..controls (1.152859,0.05) and (1.141667,0.027615)
 ..(1.141667,0)
 ..controls (1.141667,-0.027615) and (1.152859,-0.05)
 ..(1.166667,-0.05)
 ..controls (1.180474,-0.05) and (1.191667,-0.027615)
 ..(1.191667,0)
 ..controls (1.191667,0.027615) and (1.180474,0.05)
 ..(1.166667,0.05)\dpicstop
\dpicdraw (0.5,0)
 --(0.333333,0)\dpicstop
\dpicdraw (0.333333,0)
 --(0.208333,-0.125)\dpicstop
\filldraw[line width=0bp](0.221268,-0.018447)
 --(0.237529,-0.01371)
 ..controls (0.227538,0.000674) and (0.219175,0.016123)
 ..(0.212593,0.032352)
 ..controls (0.207874,0.014206) and (0.205327,-0.004436)
 ..(0.205006,-0.023183)
 --(0.221268,-0.018447)\dpicstop
\dpicdraw (0.217461,-0.002327)
 ..controls (0.225975,-0.041593) and (0.244342,-0.078046)
 ..(0.270833,-0.108253)\dpicstop
\dpicdraw (0.166667,0)
 --(0,0)\dpicstop
\end{tikzpicture}
\caption{Una resistencia de carga $R_l$ acoplada a una l�nea de transmisi�n sin p�rdidas conectada a una fuente de $CC$. Los gr�ficos debajo del circuito muestran la evoluci�n del transitorio: la primera onda incidente $v_{+1}$, la primera reflexi�n en la carga $v_{-1}$ y la segunda onda incidente $v_{+2}$. Se supone que $Rg>Z_0$ y $R_l>Z_0$.}
\label{fig:LTxSinPerFueCCCarResTra-ppKtikz}
\end{figure}

\end{document}
